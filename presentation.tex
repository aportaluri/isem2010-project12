\documentclass[utf8,color={usenames},hyperref={pdftitle={ISEM presentation},pdfauthor={Pavel Zorin}}]{beamer}
\mode<presentation>

%\usepackage[charter]{mathdesign}
\usepackage[T1]{fontenc}
%\usepackage{pzorin}

% Theorem-style
\theoremstyle{plain}
\newtheorem{thm}{Theorem}
\newtheorem{prop}[thm]{Proposition}
\theoremstyle{remark}
\newtheorem*{remark}{Remark}

\title{Mountain pass geometry}
\author[Pavel]{Pavel Zorin-Kranich \newline \texttt{pavel.zorin@online.de}}
%\href{mailto:pavel.zorin@online.de}{pavel.zorin@online.de}}
\date[ISEM'10]{13th ISEM 2010, Kacov, CZE}

\newcommand{\dist}{\mathrm{dist}}
\newcommand{\R}{\mathbb{R}}

\begin{document}

\begin{frame}
  \titlepage
\end{frame}

\begin{frame}
\frametitle{Outline}
\tableofcontents
\end{frame}

\section{Variational problem}
\begin{frame}
\frametitle{The model problem}
\begin{align*}
\tilde V(v) &= \frac14 v^4 + v^3 + v^2\\
0 &= D^4 v + \beta D^2 v + V'(v)
\end{align*}
The critical points of the Lagrange functional
\[
\tilde J_\beta(v) = \int_\R |v''|^2 / 2 - \beta |v'|^2 / 2 + \tilde V(v),
\quad
v \in H^2(\R),
\]
are characterized by
\[
0 = \left\langle \tilde J_\beta'(v), \phi \right\rangle = \int_\R v'' \phi'' - \beta v' \phi' + \tilde V'(v) \phi
\]
for every $\phi \in H^2$ and are solutions of the differential equation.
\end{frame}

\subsection{Regularity of the solutions}
\begin{frame}
\frametitle{The finite difference (``discrete differentiation'') operator}
\begin{definition}
The \emph{shift} operator:
\[
\tau_h u(x) = u(x+h).
\]
The \emph{finite difference} operator:
\[
\Delta_h u(x) = \frac{u(x+h) - u(x)}{h}.
\]
\end{definition}
Elementary properties:
\begin{align*}
\Delta_h(v u) &= (\tau_h v) \Delta_h u + (\Delta_h v) u & \text{(Leibnitz rule),}\\
\int_\R v \Delta_h u &= - \int_\R (\Delta_{-h} v) u & \text{(Partial integration)}.
\end{align*}
\end{frame}

\begin{frame}
\frametitle{Basic tool for proving differentiability}
\begin{lemma}
Let $u \in L^2_{loc}(\R)$ be such that for any bounded interval $\Omega$
\[
\int_\Omega |\Delta_h u|^2 \leq C_\Omega \quad \text{for every } h>0.
\]
Then $u' \in L^2_{loc}$.
\end{lemma}
\begin{proof}
As smooth test functions $\phi$ are dense in $L^2(\Omega)$, it suffices to observe
\begin{multline*}
\left| \int_\Omega u' \phi \right|
=
\left| \int_\Omega u \phi' \right|
=
\left| \lim_{h \to 0} \int_\Omega u (\Delta_{-h} \phi) \right|
\\=
\lim_{h \to 0} \left| \int_\Omega (\Delta_{h} u) \phi \right|
\leq C_\Omega ||\phi||_2.
\qedhere
\end{multline*}
\end{proof}
\end{frame}

\begin{frame}
\frametitle{Regularity of solutions of the variational problem}
\[
0 = \int_\R v'' \phi'' - \beta v' \phi' + \tilde V'(v) \phi
\]
Observe that $f = \tilde V'(v) \in L^2(\R)$ by the Sobolev embedding theorem.

Substitute $\Delta_{-h} \phi$ for $\phi$ and use partial integration:
\[
\int_\R (\Delta_h v'') \phi'' - \underbrace{\beta (\Delta_h v') \phi'}_{\uncover<2->{\text{unimportant}}} + f (\Delta_{-h} \phi) = 0.
\]
\pause
Specialize to $\phi = \eta^4 \Delta_h v$, where $\eta$ is a smooth positive function with compact support which is $1$ on some interval $\Omega$ such that $|\eta'| < 1$.
\end{frame}

\begin{frame}
\frametitle{Regularity of solutions of the simplified problem}
\[
\int_\R (\Delta_h v'') (\eta^4 \Delta_h v)'' + f (\Delta_{-h} (\eta^4 \Delta_h v)) = 0.
\]
\begin{align*}
\int \eta^4 |\Delta_h v''|^2
&\lesssim
\int \eta^2 |\Delta_h v''| |\Delta_h v' + \Delta_h v|\\
&+
\int |f| (\Delta_{-h}\Delta_h v)| + \int |f| |\Delta_h v |.
\end{align*}
\pause
Use the elementary inequality $2AB \leq \epsilon A^2 + B^2/\epsilon$. For example,
\[
\int \eta^2 |\Delta_h v''| |\Delta_h v'|
\leq
\epsilon \int \eta^4 |\Delta_h v''|^2 + \frac1\epsilon \int |\Delta_h v|^2.
\]
\pause
If $\epsilon$ is small enough, the first term is absorbed on the left. Procedding similarly,
\[
\int \eta^4 |\Delta_h v''|^2
\lesssim
\int |\Delta_h v'|^2
+
\int |\Delta_h v|^2
+
\int |f|^2
+
\int |\Delta_{-h}\Delta_h v|^2.
\]
\end{frame}

\begin{frame}
\frametitle{$L^2$ estimate for the finite difference operator}
\newcommand{\dif}{\text{d}}
Finite difference operator is bounded by the differentiation operator in norm:
\begin{align*}
\int_\R |(\Delta_h u)(x)|^2 \dif x
&=
\int_\R \left|
\frac{1}{h} \int_{y=x}^{x+h} u'(y) \dif y
\right|^2 \dif x\\
&\leq
\int_\R \left|
\frac{1}{h} \left(\int_{y=x}^{x+h} |u'(y)|^2 \dif y \right)^{1/2} h^{1/2}
\right|^2 \dif x\\
&=
\frac{1}{h}
\int_\R
\int_{y=x}^{x+h} |u'(y)|^2 \dif y
\dif x\\
&=
\int_\R |u'(x)|^2 \dif x
\end{align*}
\pause
\[
\int \eta^4 |\Delta_h v''|^2
\lesssim
\int |v''|^2
+
\int |v'|^2
+
\int |f|^2
=
\text{constant independent of } h
\]
By the regularity lemma, $v''' \in L^2$.
\end{frame}

\begin{frame}
\frametitle{Results so far}
We have shown that a critical point of
\[
\tilde J_\beta(v) = \int_\R |v''|^2 / 2 - \beta |v'|^2 / 2 + \tilde V(v),
\quad
v \in H^2(\R)
\]
is in fact smooth and is a classical solution of
\begin{align*}
\tilde V(v) &= \frac14 v^4 + v^3 + v^2\\
0 &= D^4 v + \beta D^2 v + V'(v).
\end{align*}
\end{frame}

\section{Mountain pass geometry}
\begin{frame}
\frametitle{Outline}
\tableofcontents[currentsection]
\end{frame}

\begin{frame}
\frametitle{Equilibrium point: a mountain pass}
\includegraphics[width=0.9\textwidth]<1>{images/mountain_pass_1.jpg}
\includegraphics[width=0.9\textwidth]<2>{images/mountain_pass_2.jpg}
\includegraphics[width=0.9\textwidth]<3>{images/mountain_pass_3.jpg}
\includegraphics[width=0.9\textwidth]<4>{images/mountain_pass_4.jpg}
\includegraphics[width=0.9\textwidth]<5>{images/mountain_pass_5.jpg}
\end{frame}

\begin{frame}
\frametitle{Existence of an approximate mountain pass}
\begin{theorem}
Let $H$ be a Hilbert space, $h \in C^2(H)$, $x_0,x_1 \in H$ such that $h(x_0) = h(x_1) = 0$, but
\begin{equation}
\inf_\gamma \max_t h(\gamma(t)) = H > 0,
\end{equation}
the infimum being taken over continuous paths $\gamma$ from $x_0$ to $x_1$.

Then there exist points $x$ with, simultaneously, $h(x)$ arbitrarily close to $H$ and $h'(x)$ arbitrarily close to $0$.
\end{theorem}
\pause
\begin{proof}
Assume not. Then, in some region $\{ |h-H| < 2 \epsilon \}$, $|h'|$ is bounded from below by $\epsilon$. Take a path $\gamma$ such that
\[
\max_t h(\gamma(t)) < H + \epsilon.
\]
The next lemma allows to construct a path $\tilde\gamma$ with maximal height smaller then $H$, a contradiction.
\end{proof}
\end{frame}

%\subsection{Deformation lemma: existence of steepest descent flow}
\begin{frame}
\frametitle{Deformation lemma}
\begin{lemma}
Let $H$ be a Hilbert space
and $h \in C^1 (H)$, $h' \in \text{Lip}_\text{loc}$ be the height function.

Assume $|h'|>\epsilon$ in $\{ |h - H| \leq 2 \epsilon \}$.

Then there exists a flow $\eta : [0,2 \epsilon] \times H \to H$ which
\begin{itemize}
\item leaves $\{ |h - a| > 2 \epsilon \}$ fixed,
\item after time $2\epsilon$, maps $\{ h < a + \epsilon\}$ into $\{ h \leq a - \epsilon \}$.
\end{itemize}
\end{lemma}
\end{frame}

\begin{frame}
\frametitle{Deformation lemma, proof}
\includegraphics[width=0.5\textwidth]<1>{images/mountain_pass_qd_0.jpg}
\includegraphics[width=0.5\textwidth]<2>{images/mountain_pass_qd_2.jpg}
\includegraphics[width=0.5\textwidth]<3>{images/mountain_pass_qd_3.jpg}

\begin{itemize}
\item<2-> Cutoff function $g(x) := \frac{\dist(x,\complement A)}{\dist(x,\complement A) + \dist(x,B)}$.
\begin{itemize}
\item $g|_{\color{magenta} B} \equiv 1$, $g|_{\color{red} \complement A} \equiv 0$ and $g$ is locally Lipschitz.
\end{itemize}

\item<3-> Locally Lipschitz vector field ${\color{blue}X}(x) = - \frac{h'(x)}{|h'(x)|^2} g(x)$.
\begin{itemize}
\item Picard-Lindelöf Theorem $\implies$ $X$ defines a unique continuous flow $\eta$.
\item $X$ bounded $\implies$ the flow $\eta$ exists for all times.
\item ${\color{blue} X} h|_{\color{magenta} B} = -1$.
\end{itemize}

\item<3-> Thus the time $2 \epsilon$ map of the flow $\eta$ maps $B$ into $\{ h \leq a - \epsilon \}$.
\end{itemize}
\end{frame}


\begin{frame}
\frametitle{Deformation lemma, local version}
\begin{lemma}
Let $H$ be a Hilbert space
and $h \in C^1 (H)$, $h' \in \text{Lip}_\text{loc}$ be the height function.
\uncover<2->{Let {\color{red} $S \subset H$}.}

Assume $|h'|>\epsilon$ in $\{ |h - H| \leq 2 \epsilon \}$
\uncover<2->{{\color{red} $\cap \{ \dist (\cdot,S) \leq 4 \}$}}

Then there exists a flow $\eta : [0,2 \epsilon] \times H \to H$ which
\begin{itemize}
\item leaves $\{ |h - a| > 2 \epsilon \}$
\uncover<2->{{\color{red} and $\{ \dist (\cdot,S) > 4 \}$}}
fixed,
\item after time $2\epsilon$, maps $\{ h < a + \epsilon\}$
\uncover<2->{{\color{red} $\cap S$}}
into $\{ h \leq a - \epsilon \}$.
\end{itemize}
\end{lemma}

\uncover<3->{
\begin{proof}
Use a cutoff function which satisfies
\[
\phi(x) =
\begin{cases}
1, & \dist(x, S) \leq 2\\
%2 - \dist(x, S)/2, & 2 < \dist(x, S) < 4\\
0, & \dist(x, S) \geq 4.
\end{cases}
\]
Proceed as before, but use the vector field $\phi X$.
As the vector field is uniformly bounded, the flow lines starting in $S$ will stay in $\{\dist(\cdot, S) \leq 2\}$ long enough, so that $\phi X h = -1$ where needed.
\end{proof}
}
\end{frame}

\section{Lagrange functional with mountain pass structure}
\begin{frame}
\frametitle{Outline}
\tableofcontents[currentsection]
\end{frame}

\begin{frame}
\frametitle{A version of the Lagrange functional with mountain pass structure}
$\tilde J_\beta$ doesn't have the mountain pass structure.

To overcome this difficulty, we will consider a cut-off potential and the corresponding functional
\begin{align*}
V(v) &=
\begin{minipage}{6cm}
$\begin{cases}
v^4 / 4 + v^3 + v^2 & v \geq -2\\
0 & v < -2,
\end{cases}$
\end{minipage}
\\
J_\beta(v) &= \int_\R |v''|^2 / 2 - \beta |v'|^2 / 2 + V(v).
\end{align*}
A critical point $v$ of $J_\beta$ is also a critical point of $\tilde J_\beta$ if only it satisfies the pointwise inequality $v>-2$. In such a case it is also a classical solution of our equation.
\end{frame}


\begin{frame}
\frametitle{Mountain pass structure of $J_\beta$: valley at $0$}
We work in a fixed range $0 < \beta_{min} < \beta < \beta_{max} < \sqrt 8$.
\begin{lemma}
For some $\epsilon > 0$, $\delta > 0$
\[
J_\beta(v) \geq \epsilon ||v||_{H^2}^2, \quad \text{whenever } ||v|| < \delta.
\]
\end{lemma}
\begin{proof}
Sobolev embedding theorem and a calculation.
\end{proof}
\end{frame}

\begin{frame}
\frametitle{Mountain pass structure of $J_\beta$: a second valley}
\begin{lemma}
There exists $e \in H^2(R)$ such that $J_\beta(e)<0$ for every $\beta \in [\beta_{min},\beta_{max}]$.
%\textbf{Probably no proof, just say the cut-off is useful}
\end{lemma}
\begin{proof}
Let $0 \not\equiv f$ be a negative smooth function with compact support. Its dilates $f_\lambda(x)=f(\lambda x)$ satisfy.
\[
\int (f_\lambda'')^2 \sim \lambda^3, \quad \int (f_\lambda')^2 \sim \lambda.
\]
\[
\text{Thus, for}\  \lambda>0\ \text{small,} \quad
j_\beta = \frac12 \int_\R (f_\lambda'')^2 - \beta (f_\lambda')^2 < 0.
\]
As the support of $f_\lambda$ is bounded and $V(v)$ is bounded for $v\leq 0$,
\[
J_\beta(C f_\lambda) = j_\beta C^2 + \text{const}, \quad
\text{(this is where the cut-off comes in!)}
\]
Thus, for $C$ big enough, $e = C f_\lambda$ has the required properties.
\end{proof}
\end{frame}

\begin{frame}
\frametitle{Pass levels}
Define the pass levels by
\[
H_\beta = \inf_\gamma \sup_{t} J_\beta(\gamma(t)),
\]
where the infimum is taken over continuous paths $\gamma$ from $0$ to $e$.

%Such paths shall be called \emph{admissible.}

\pause
Recall
\[
J_\beta(v) = \text{(something)} - \beta \text{(something positive)}.
\]

Thus the pass level function $H_\beta$ is monotonously decreasing and therefore differentiable almost everywhere.
\end{frame}

\begin{frame}
\frametitle{Minimizing sequence for $J_\beta'$}
\begin{lemma}
Let $\beta \in [\beta_{min}, \beta_{max}]$ be such that the derivative $H'_\beta$ exists. Then there exists a sequence $(v_n) \subset H^2(R)$ which satisfies
\begin{enumerate}
\item $J_\beta(v_n) = H_\beta + o(1)$,
\item $J_\beta'(v_n) = o(1)$,
\item $\frac12 ||v_n'||_2^2 \leq -2H'_\beta + 2$.
\end{enumerate}
\end{lemma}
\pause
\emph{Proof}

Define the admissible region
\[
S = \{ u\ :\ \frac12 || u' ||_2^2 \leq - H'_\beta + \frac12 \}
\]
Assume the opposite. Then there exists an $\epsilon <  H_\beta/2$ such that
\[
||J'_\beta|| > 32 \epsilon
\, \text{on} \,
\{ |J_\beta - H_\beta| < 2 \epsilon \} \cap \{ \dist(\cdot,S) \leq 1/2 \}.
\]
\end{frame}

\begin{frame}
\frametitle{Minimizing sequence for $J_\beta'$: flow decreasing $J_\beta$}
By the local deformation lemma there exists a flow $\eta$ such that
\begin{itemize}
\item $\{ |J_\beta - H_\beta| \geq 2 \epsilon \}$ and $\{ \dist(\cdot,S) > 1/2 \}$ are let fixed,
\begin{itemize}
\item<2-> {\color{red} $0$ and $e$ are fixed}
\end{itemize}
\item $J_\beta$ decreases along $\eta$,
\item time $2\epsilon$ map maps $\{ J_\beta \leq H_\beta + \epsilon\} \cap S$ into $\{J_\beta \leq H_\beta - \epsilon\}$.
\end{itemize}
\uncover<2->{
For a contradiction, look for a path {\color{red} $\gamma$ from $0$ to $e$} such that
\begin{itemize}
\item $\gamma \subset S$,
\item $J_\beta(\gamma) \leq H_\beta + \epsilon$.
\end{itemize}
}
\end{frame}

\begin{frame}
\frametitle{Minimizing sequence for $J_\beta'$: using differentiability of the pass level function}
Choose $H_{\tilde\beta} < H_\beta$ later
\uncover<2->{{\color{red} using differentiability of $H_\beta$ in $\beta$.}}

By definition of $H_{\tilde\beta}$, there is a path $\tilde\gamma$ such that, for all $t$,
\[
\uncover<3->{{\color{blue} J_\beta(\tilde\gamma(t)) \leq}}
J_{\tilde\beta} (\tilde\gamma(t)) \leq H_{\tilde\beta} + (\beta - \tilde\beta)/8
\uncover<2->{{\color{red} \leq H_\beta' + \epsilon}}
\]
For every $t$ with $J_\beta(\tilde\gamma(t)) \geq H_\beta - (\beta-\tilde\beta)/8$, one has
\[
\frac12 ||\tilde\gamma(t)'||_2^2
=
\frac{J_{\tilde\beta}(\tilde\gamma(t)) - J_\beta(\tilde\gamma(t))}{\beta - \tilde\beta}
\leq
\frac{H_{\tilde\beta} - H_\beta}{\beta - \tilde\beta} + \frac14
\uncover<2->{{\color{red} \leq - H'_\beta + \frac12,}}
\]
\uncover<3->{i.e. {\color{blue} $\tilde\gamma \subset S = \{ u :\, \frac12 || u' ||_2^2 \leq - H'_\beta + \frac12 \}$}}

\uncover<4->{{\color{blue} Local deformation lemma: $\gamma = \eta_{2 \epsilon} (\tilde\gamma)$} is a path from $0$ to $e$ and
\[
J_\beta(\gamma) \leq H_\beta - \epsilon \text{ -- contradiction to the definition of } H_\beta.
\qedhere
\]
}
\end{frame}

\begin{frame}
\frametitle{Non-triviality of the weak limit}
The sequence $v_n$ cannot converge to $0$ uniformly. Wlog one can assume that it is bounded away from $0$ in the supremum norm.
Assume
\[
\limsup_n (\min_x v_n(x)) \geq -1.
\]
In such a case, using $V(v) \geq v^2/4$, one obtains that $v_n$ is bounded in $H^2$.

Shift the $v_n$ such that the shifted versions have a maximum at $0$:
\[
w_n := \tau_{d_n} v_n, \quad d_n \ \text{such that}\  |v_n(d_n)| = \max_x |v_n(x)|.
\]
Thus, if $w_n \rightharpoonup w$, then $w(0) \neq 0$, as point evaluations are weakly continuous. It follows that $w$ is a critical point of $J_\beta$.
\end{frame}

\end{document}

%<!-- Local IspellDict: american -->

\documentclass[pdftex,10pt,intlimits]{beamer}
%\documentclass[pdftex,10pt]{beamer}
%\documentclass[pdftex,10pt,hyperref={pdfpagemode=FullScreen,pdfstartpage={1}}]{beamer}

\mode<presentation>
{
  \usetheme{Warsaw}
  % or ...
  \setbeamercovered{transparent}
  % or whatever (possibly just delete it)
}

%\usepackage{eufrak}%packages for fractional fonts

%\usepackage[notcite,notref]{showkeys}
%\usepackage[active]{srcltx}%inverse search

\usepackage{graphicx}
%\hypersetup{%
%  pdftitle={},
%  pdfsubject={},
%  pdfauthor={},
%  pdfkeywords={},
%  pdfcreator={pdfLaTeX},
%  pdfproducer={pdfLaTeX},
%  }
%\usepackage{eepic}
\usepackage{color}

\usepackage{amsmath}%center-tags centered tags at split
\usepackage{amsthm}
\usepackage{amssymb}
\usepackage{amsfonts}
\usepackage{amsxtra}

% \usepackage{enumerate}

\usepackage{ifthen} %
\usepackage{xspace} %


\usepackage{paper_diening}

%% symbol for mean value
\newcommand{\meantmp}[2]{#1\langle{#2}#1\rangle}
\newcommand{\mean}[1]{\meantmp{}{#1}}
\newcommand{\bigmean}[1]{\meantmp{\big}{#1}}
\newcommand{\Bigmean}[1]{\meantmp{\Big}{#1}}
\newcommand{\biggmean}[1]{\meantmp{\bigg}{#1}}
\newcommand{\Biggmean}[1]{\meantmp{\Bigg}{#1}}


%\numberwithin{equation}{section}

% %own parameters
% \newlength{\boxparameter}
% \setlength{\boxparameter}{\textwidth}
% \addtolength{\boxparameter}{-6pt}

\definecolor{lightgrey}{rgb}{.7,.7,.7}
\newcommand{\red}[1]{\textcolor{red}{#1}}
\newcommand{\blue}[1]{\textcolor{blue}{#1}}
\newcommand{\hellgrau}[1]{\textcolor{lightgrey}{#1}}


% hyphenation list------------------------------------------------
\hyphenation{ }

\title{The Kawahara Equation}

% abstract: It has been possible in the recent years to generalize parts
% of the linear Calderon-Zygmund theory to the non-linear setting of the
% p-Laplacian. We discuss the basic principles of this method and
% generalize it to the context of Non-Newtonian fluids.

\author{ \inst{1} \and }%
% - Give the names in the same order as the appear in the paper.
% - Use the \inst{?} command only if the authors have different
%   affiliation.

\institute
{
  \inst{1}%
}
% - Use the \inst command only if there are several affiliations.
% - Keep it simple, no one is interested in your street address.



\date{May 19, 2010}
% - Either use conference name or its abbreviation.
% - Not really informative to the audience, more for people (including
%   yourself) who are reading the slides online


% If you have a file called "university-logo-filename.xxx", where xxx
% is a graphic format that can be processed by latex or pdflatex,
% resp., then you can add a logo as follows:
%\pgfdeclareimage[height=0.5cm]{university-logo}{unilogo}
%\logo{\pgfuseimage{university-logo}}
%\titlegraphic{\includegraphics[height=1cm]{unilogo}\hspace*{1cm}\includegraphics[height=1cm]{iam}}




% Delete this, if you do not want the table of contents to pop up at
% the beginning of each subsection:
% \AtBeginSubsection[]
% {
%   \begin{frame}<beamer>
%     \frametitle{Outline}
%     \tableofcontents[currentsection,currentsubsection]
%   \end{frame}
% }


% If you wish to uncover everything in a step-wise fashion, uncomment
% the following command:

%\beamerdefaultoverlayspecification{<+->}

\begin{document}

\begin{frame}
 \titlepage
 % You might wish to add the option [pausesections]
\end{frame}


\section{Introduction}

\begin{frame}
  \frametitle{The Kawahara Equation}
  %\framesubtitle{Optional}
  \begin{block}{}
  $$u_t + u_x + u_{xxx} + u_{xxxxx} + (u^{q+1})_x = 0 \mbox{ for } q\ge 1$$
  \end{block}
  \begin{itemize}
   \item Higher order generalization of famous Korteweg-de Vries equation:\\ \hspace{2.5cm} $u_t + u_{xxx} + uu_x = 0$
   \item A model for (among others) propagation of low amplitude water waves in shallow canal, subject to gravity and
capillarity
   \item Hamiltonian system: solutions are orbits along the {\bf symplectic} gradient of
energy $H$, ``symplectic gradient system''\\
$\leadsto$ Conservation of energy, instead of dissipation
  \end{itemize}
\begin{block}{}
{\begin{bf} Reminder\end{bf}}: Vector space $V$, symplectic form $\omega$ on $V$, i.e.\\
\vspace{0.15cm}
\hspace{0.5cm} nondegenerate, skew-symmetric bilinear form (and $d\omega=0$),\\
\vspace{0.15cm} functional $H: V\rightarrow \set R$. Define
$\omega(X_H,\cdot) := dH$.
$$\leadsto u_t=X_H(u)$$
\end{block}
\end{frame}

\begin{frame}
  \frametitle{Nonlinear Dispersive Equations}
  %\framesubtitle{Optional}
Linear (nondissipative) wave type equations usually exhibit  {\bf dispersion}:\\
\vspace{0.15cm}
\hspace{0.25cm} Different Fourier components of a solution travel at different speeds.\\
\vspace{0.15cm}
\begin{block}{Example}
$u(t,x)=Ae^{i(t\omega + xk)}$ solves Airy equation $u_t + u_{xxx} = 0$ iff $\omega =
k^3$, i.e.$$u(t,x)=Ae^{ik(x -  (-k^2)t)}$$
\vspace{-0.5cm}
\begin{itemize}
 \item All frequencies propagate leftward
 \item Higher frequencies propagate faster than lower ones
\end{itemize}
$\leadsto$ Localised high-amplitude states disperse leftward into broader, lower amplitude states (but conserving
energy!)
\end{block}
\vspace{0.25cm}
{\bf But}: Solitary waves (solitons) can be observed in nature. $\leadsto$ Nonlinearity counteracts
dispersion
\end{frame}

\begin{frame}
  \frametitle{Nonlinear Dispersive Equations}
  %\framesubtitle{Optional}
Some remarkable properties of KdV:
\begin{itemize}
 \item admits stable solitary wave solutions moving to the right, while nonlocal states move to the left
 \item colliding solitons interact nonlinearly, but emerge almost unchanged (shifted)
 \item completely integrable, roughly: infinitely many conserved quantities $\leadsto$ phase space (formally) reduces to
a point
 \item inverse scattering technique: (almost) explicit formula for solution to arbitray (sufficiently regular) initial
data
 \item generic initial data: most of the solution ``radiates'' away to the left, while finite number of solitons moving
to the right emerges
\end{itemize}
Same approach possible for gKdV, $u_t + u_{xxx} + (u^q)_x = 0$ with $q=3$, and some NLS equations, {\bf but
not} for more general equations like gKdV with $q\not=2,3$ or the Kawahara equation.
\end{frame}

\section{Physical Derivation of the Kawahara Equation}

\begin{frame}
  \frametitle{2d-Euler equations}
  %\framesubtitle{Optional}
  Ideal, inviscid, incompressible, irrotational two-dimensional fluid with free boundary subject to capillarity and
gravity. After non-dimensionalisation:
\begin{block}{}
\begin{equation*}\begin{aligned}
  u_t + \alpha(uu_x + wu_z) + p_x &= 0,\\
 \beta [w_t + \alpha(uw_x + ww_z)] + p_z &=0,\\
 \Delta \phi &= 0,
\end{aligned}\end{equation*}
\end{block}
with velocity field $(u,w)=\nabla\phi$ in $(x,z)$ coordinates, ``pressure'' $p$, $\alpha=a/h_0$, $\beta=(h_0/l)^2$,
amplitude and wavelength $a,l$ of the surface wave, height $h_0$ of the undisturbed surface. Boundary conditions:
\begin{block}{}
\begin{equation*}\begin{aligned}
  w&=0 \mbox{ at } z=0,\\
  w&=\eta_t+\alpha u\eta_x\mbox{ at } z=1+\alpha\eta,\\
  p&=\eta - \beta\tau\frac{\eta_{xx}}{(1+\alpha^2\beta\eta_x^2)^{3/2}}\mbox{ at } z=1+\alpha\eta,
\end{aligned}\end{equation*}
\end{block}
with free surface $h=1 + \alpha\eta$, Bond number $\tau=\Gamma/\rho g h_0^2$, surface tension coefficient $\Gamma$,
density $\rho$, gravitional acceler. $g$.
\end{frame}

\begin{frame}
  \frametitle{Coupling of $\alpha$ and $\beta$}
  %\framesubtitle{Optional}
  {\bf Goal}: Eliminate $u,w,p$ in the small-amplitude and long-wavelength limit.\\
 \bigskip
Linearisation at $0$ and plugging $\eta=Ae^{i(\omega t - kx)}$ gives
\begin{block}{}{\small
\begin{equation*}\begin{aligned}
\omega^2 = \frac{g}{h_0}\Big((1+\tau k^2h_0^2)kh_0\tanh(kh_0)\Big) \approx gh_0k^2\Big[1-\Big(\frac{1}{3} -
\tau\Big)k^2h_0^2 + \mbox{h.o.t.}\Big]
\end{aligned}\end{equation*}}
\end{block}
in the long wavelength limit $kh_0 \ll 1$.\\
\bigskip
$\leadsto$ Leading order dispersive term is of order $(\frac{1}{3}-\tau)k^2h_0^2$, equiv.
$(\frac{1}{3}-\tau)\beta$
\begin{block}{}
Since the nonlinearity is of order $\alpha$, we need
\begin{equation*}\begin{aligned}
  \alpha=O(\beta) &\mbox{ if } \abs{1/3-\tau}=O(1),\\
  \alpha=O(\beta^2) &\mbox{ if } \abs{1/3-\tau}=O(\beta).
\end{aligned}\end{equation*}
\end{block}
\end{frame}

\begin{frame}
  \frametitle{Eliminating $u,w,p$}
  %\framesubtitle{Optional}
  Course of action:
  \begin{itemize}
   \item Express the boundary conditions by Taylor series expansion at $z=1$:
\begin{block}{}
\vspace{-0.35cm}
\begin{equation*}\begin{aligned}
  w + \alpha\eta w_z + \frac{\alpha^2\eta^2}{2} w_{zz}&=\eta_t+\alpha\eta_x (u + \alpha\eta u_z) + O(\alpha^3) \mbox{ at
} z=1,\\
  p + \alpha\eta p_z + \frac{\alpha^2\eta^2}{2}p_{zz}&=\eta
- \beta\tau\eta_{xx} + O(\alpha^3,\alpha^2\beta^2)\mbox{ at } z=1,
\end{aligned}\end{equation*}
\end{block}
\item Expand $u,w,p,\eta$ into double power series $q=\sum_{i,j=0}^\infty \alpha^i\beta^jq_{ij}$
\item Plug expansions into Euler equations and expanded boundary conditions
\item Compare the coefficients of terms of order $O(1),O(\alpha),O(\beta),O(\alpha^2),O(\beta^2),O(\alpha\beta)$
\item Combine to get equations for $\eta_{ij}$
 \end{itemize}
\end{frame}

\begin{frame}
  \frametitle{Equations for $\eta_{ij}$}
  %\framesubtitle{Optional}
{\small\begin{equation*}\begin{aligned}
\eta_{00tt} - \eta_{00xx} &= 0\\
\eta_{10tt} - \eta_{10xx} &= \Big[\frac{\eta_{00}^2}{2} +
\Big(\int_{-\infty}^x\eta_{00t}\Big)^2\Big]_{xx}\\
\eta_{01tt} - \eta_{01xx} &= \Big(\frac{1}{3}-\tau\Big)\eta_{00xxxx}\\
\eta_{20tt} - \eta_{20xx} &= (\eta_{00}\eta_{10})_{xx} + 2
\Big[\int_{-\infty}^x\eta_{00t}\int_{-\infty}^x\eta_{10t}\Big]_{xx} -
\Big[\eta_{00}\Big(\int_{-\infty}^x\eta_{00t}\Big)^2\Big]_{xx}\\
\eta_{02tt} - \eta_{02xx} &= \Big(\frac{1}{3}-\tau\Big)\eta_{01xxxx} +
\frac{1}{3}\Big(\frac{2}{5}-\tau\Big)\eta_{00xxxxxx}\\
\eta_{11tt} - \eta_{11xx} &= 2\Big(\int_{-\infty}^x\eta_{00t}\int_{-\infty}^x\eta_{01t}\Big)_{xx} +
\Big(\frac{1}{3}-\tau\Big)\eta_{10xxxx} + \frac{2}{3}(\eta_{00t}^2)_{xx} + (\eta_{00}\eta_{00xx})_{xx}\\
& \hspace{0.25cm}- \tau(\eta_{00}\eta_{00xxx})_x
\end{aligned}\end{equation*}}
\red{Combine according to $\eta=\sum_{i,j=0}^\infty \alpha^i\beta^j\eta_{ij}$.}
\end{frame}

% \begin{frame}
%   \frametitle{Approximate equation for $\eta$}
%   %\framesubtitle{Optional}
% \begin{block}{Boussinesq type equation}
% {\begin{equation*}\begin{aligned}
% \eta_{tt}-\eta_{xx} &-\alpha\Big[\frac{\eta^2}{2} + \Big(\int_{-\infty}^x\eta_t\Big)^2 \Big]_{xx} - \beta
% \Big[\frac{1}{3} - \tau\Big]\eta_{xxxx} + \alpha^2\Big[\eta\Big(\int_{-\infty}^x\eta_t\Big)^2\Big]_{xx} \\
% &+ \alpha\beta \Big[\frac{2}{3}(\eta_t^2)_{xx} + (\eta\eta_{xx})_{xx} -\tau(\eta\eta_{xxx})_x\Big] -
% \frac{\beta^2}{3}\Big[\frac{2}{5}-\tau\Big]\eta_{xxxxxx} = 0
% \end{aligned}\end{equation*}}
% \end{block}
% \begin{itemize}
% \item Models bi-directional propagation of small-amplitude long-\\ wavelength capillarity-gravity water waves up to
% order $O(\alpha^2)$,\\ $O(\beta^2),O(\alpha\beta)$
% \item Nonlocal in space
% \item Validity not uniform in $t$ due to growth of $\eta_{ij}$
% \item Linearly well-posed for $\tau<2/5$
% \item Nonlinearity of order $O(\alpha),O(\alpha^2),O(\alpha\beta)$
% \item Dispersion of order $O(\beta),O(\beta^2)$
% \end{itemize}
% \end{frame}

\begin{frame}
  \frametitle{Approximate equation for $\eta$}
  %\framesubtitle{Optional}
\begin{block}{Boussinesq type equation}
{\begin{equation*}\begin{aligned}
\eta_{tt}-\eta_{xx} &-\alpha\Big[\frac{\eta^2}{2} + \Big(\int_{-\infty}^x\eta_t\Big)^2 \Big]_{xx} - \beta
\Big[\frac{1}{3} - \tau\Big]\eta_{xxxx} + \alpha^2\Big[\eta\Big(\int_{-\infty}^x\eta_t\Big)^2\Big]_{xx} \\
&+ \alpha\beta \Big[\frac{2}{3}(\eta_t^2)_{xx} + (\eta\eta_{xx})_{xx} -\tau(\eta\eta_{xxx})_x\Big] -
\frac{\beta^2}{3}\Big[\frac{2}{5}-\tau\Big]\eta_{xxxxxx} = 0
\end{aligned}\end{equation*}}
\end{block}
\begin{itemize}
\item Taken into account terms of order up to $O(\alpha^2)$, $O(\beta^2),O(\alpha\beta)$
\item Validity not uniformly in $t$ due to growth of $\eta_{ij}$
\item Nonlinearity of order $O(\alpha),O(\alpha^2),O(\alpha\beta)$
\item Dispersion of order $O(\beta),O(\beta^2)$
\end{itemize}
\end{frame}

\begin{frame}
  \frametitle{Coupling of $\alpha$ and $\beta$}
  %\framesubtitle{Optional}
\begin{block}{Case 1: $\abs{1/3-\tau}\gg\beta$}
We need $\alpha=O(\beta)$. Up to order $O(\alpha)=O(\beta)$ we get
\begin{equation*}\begin{aligned}
\eta_{tt}-\eta_{xx} &-\alpha\Big[\frac{\eta^2}{2} + \Big(\int_{-\infty}^x\eta_t\Big)^2 \Big]_{xx} \pm \alpha c_1
\eta_{xxxx} = 0
\end{aligned}\end{equation*}
\end{block}
\begin{block}{Case 2: $\abs{1/3-\tau}=O(\beta)$}
We need $\alpha=O(\beta^2)$. Up to order $O(\alpha)=O(\beta^2)$ we get
\begin{equation*}\begin{aligned}
\eta_{tt}-\eta_{xx} &-\alpha\Big[\frac{\eta^2}{2} + \Big(\int_{-\infty}^x\eta_t\Big)^2 \Big]_{xx} \pm \alpha c_1
\eta_{xxxx} - \alpha c_2 \eta_{xxxxxx} = 0
\end{aligned}\end{equation*}
\end{block}
\end{frame}

% \begin{frame}
%   \frametitle{Boussinesq and KdV type equations}
%   %\framesubtitle{Optional}
%   \begin{itemize}
%    \item Coord. transformation: $X=\frac{1}{\sqrt{c_1}}\Big(x+\alpha\int_{-\infty}^x\eta(y,t)dy\Big),
% T=\frac{1}{\sqrt{c_1}}t$
%    \item Substitute $N=\frac{3}{2}(\eta-\alpha\eta^2)$
%    \item Neglect terms of order $O(\alpha^2)$ and higher
%   \end{itemize}
% \begin{block}{Boussinesq type equations}
% \begin{equation*}\begin{aligned}
% \mbox{Case 1: } &N_{TT} - N_{XX} - \alpha (N^2)_{XX} \pm\alpha N_{XXXX} = 0 \\
% \mbox{Case 2: } &N_{TT} - N_{XX} - \alpha (N^2)_{XX} \pm\alpha N_{XXXX} -c_3\alpha N_{XXXXXX} = 0
% \end{aligned}\end{equation*}
% \end{block}
%   \begin{itemize}
%    \item Coord. transformation: $\xi=X-T,\tau=\alpha T$
%    \item Neglect terms of order $O(\alpha^2)$ and higher
%    \item Integrate once
%    \item Scaling of coordinates
%  \end{itemize}
% \begin{block}{KdV type equations}
% \begin{equation*}\begin{aligned}
% \mbox{Case 1: } &N_\tau + NN_\xi + N_{\xi\xi\xi} = 0 \\
% \mbox{Case 2: } &N_\tau + NN_\xi + N_{\xi\xi\xi} + N_{\xi\xi\xi\xi\xi} = 0
% \end{aligned}\end{equation*}
% \end{block}
% \end{frame}

\begin{frame}
  \frametitle{Boussinesq type equations}
  %\framesubtitle{Optional}
  \begin{itemize}
   \item Coord. transformation: $X=\frac{1}{\sqrt{c_1}}\Big(x+\alpha\int_{-\infty}^x\eta(y,t)dy\Big),
T=\frac{1}{\sqrt{c_1}}t$
   \item Substitute $N=\frac{3}{2}(\eta-\alpha\eta^2)$
   \item Neglect terms of order $O(\alpha^2)$ and higher
  \end{itemize}
\begin{block}{Boussinesq type equations}
\begin{equation*}\begin{aligned}
\mbox{Case 1: } &N_{TT} - N_{XX} - \alpha (N^2)_{XX} \pm\alpha N_{XXXX} = 0 \\
\mbox{Case 2: } &N_{TT} - N_{XX} - \alpha (N^2)_{XX} \pm\alpha N_{XXXX} -c_3\alpha N_{XXXXXX} = 0
\end{aligned}\end{equation*}
\end{block}
Models for bi-directional propagation of small-amplitude long-wavelength capillarity-gravity water waves
  \begin{itemize}
   \item Case 1: positive dispersive term for $1/3\ll\tau$, negative dispersive term for $0\le\tau\ll 1/3$ ($\leadsto$
ill-posed, short-wave instability)
   \item Case 2: positive dispersive term for $\tau$ slightly smaller than $1/3$, negative for $\tau$
slightly bigger than $1/3$
 \end{itemize}
\end{frame}

\begin{frame}
\frametitle{KdV type equations}
  \begin{itemize}
   \item Coord. transformation: $\xi=X-T,\tau=\alpha T$
   \item Neglect terms of order $O(\alpha^2)$ and higher
   \item Integrate once
   \item Scaling of coordinates
 \end{itemize}
\begin{block}{KdV type equations}
\begin{equation*}\begin{aligned}
\mbox{Case 1: } &N_\tau + NN_\xi + N_{\xi\xi\xi} = 0 \\
\mbox{Case 2: } &N_\tau + NN_\xi + N_{\xi\xi\xi} + N_{\xi\xi\xi\xi\xi} = 0
\end{aligned}\end{equation*}
\end{block}
\end{frame}


\end{document}


%% ZEIT
